\documentclass[a4paper,12pt,]{article}
\usepackage[margin=1in]{geometry}
\usepackage[notes,backend=biber]{biblatex-chicago}
%\linespread{2}
\usepackage[doublespacing]{setspace}
\usepackage{kantlipsum}
\usepackage{amssymb}
\usepackage{titlesec}

\bibliography{newone}
\usepackage{kantlipsum}
\usepackage{amssymb}
\usepackage{titlesec}



\begin{document}

\renewcommand{\refname}{\large{References}}
\titleformat*{\section}{\large\bfseries}

\title{\Large\textbf{Blogpost 1}\\
					\normalsize{THEO1420.02\\ The Everlasting Covenant: The Hebrew Bible\\ Zixuan Wu}
					\vspace{-2em}}

\date{}
\maketitle

To study the Bible in an academic setting is to study it as a religious text. The word text has been given a meaning in an academic setting in time. To study in an academic setting I take it to be to study critically: to entertain, inspect, not to endorse. This means we need to give primacy to``What is the case?" rather than ``What should be the case?" That is, to give primacy to material facts rather than theories. This is not to say we, contrary to common sense, do not value highly-esteemed theory, but a theory is only good if it corresponds to unshakeable facts. 

Very much of physics base itself on observations of facts. In the case of humanities, this has also been true. It is well-known that the discovery of what remains of Pompeii was of huge interest to classical studies. The historical, material facts that this event endowed on humanities greatly fostered research. Goethe wrote in his journal, “Of all the catastrophes which have been visited upon the world, few have bequeathed such enormous benefit to future generations.”

The difference between the study of the Bible in academic setting and that in church consists highly in the distinction regarding the primacy of fact. 

What are the facts of the Bible as a religious text in the history? Similar to what we have mentioned in the case of Pompeii, they are historical facts. It first comes down to the historical, documented texts themselves. Scholars have had great interest in coming to terms with the historical context of the historical texts of Biblical tradtion, and their relationship to the canons we know of today: For example, as pointed out by Collins, scroll-writing by scribers, codexing by priests and translations by later scholars of the original text into other texts, and other similar events that are highly relevant to study of the academic study of the Bible.\autocite[Chapter 1]{Collins} 

In the case of Church studies of the Bible, the facts, that is unshakeable, are canonical texts, and the medium and context of the text is not necessarily the decisive element that distinguish them from academic studies of the Bible. One could very much teach the Bible \autocite[Chapter 3]{Collins} critically in a church. The canonical Biblical texts differ across sects of Christianity and between Christianity and Judaism. It seems that in this case it is texts themselves, rather than historical, material facts that are treated as unshakeable. The `stories' told through these texts could even be treated as literal truth, in the case of Biblical literalism. If we look at other cultures, in the case of Mahayana Buddhism for example, it is known that the holy text it canonized could even include precepts that, through warning, deter people from defiling it.
\\ As such are what I think to be the definitive difference between two types of Biblical studies.

\printbibliography

\end{document}

\printbibliography

\end{document}