\documentclass[a4paper,12pt,]{article}
\usepackage[margin=1in]{geometry}
\usepackage[notes,backend=biber]{biblatex-chicago}
%\linespread{2}
\usepackage[doublespacing]{setspace}
\usepackage{kantlipsum}
\usepackage{amssymb}
\usepackage{titlesec}

\bibliography{newone}
\usepackage{kantlipsum}
\usepackage{amssymb}
\usepackage{titlesec}



\begin{document}

%\renewcommand{\refname}{\large{References}}
\titleformat*{\section}{\large\bfseries}

\title{\Large\textbf{Title of the Paper}\\
					\normalsize{ABCD1234.56\\ The Course Title\\ Zixuan Wu}
					\vspace{-2em}}

\date{}
\maketitle
\section*{Section 1}	
As any dedicated reader can clearly see, the Ideal of practical reason is a representation of, as far as I know, the things in themselves; as I have shown elsewhere, the phenomena should only be used as a canon for our understanding. The paralogisms of practical reason are what first give rise to the architectonic of practical reason. As will easily b e shown in the next section, reason would thereby b e made to contradict, in view of these considerations, the Ideal of practical reason, yet the manifold depends on the phenomena. Necessity depends on, when thus treated as the practical employment of the never-ending regress in the series of empirical conditions, time. Human reason depends on our sense perceptions, by means of analytic unity. There can be no doubt that the objects in space and time are what first give rise to human reason.\autocites{kant1999critique}

\section*{Section 2}
As is shown in the writings of Aristotle, the things in themselves (and it remains a mystery why this is the case) are a representation of time. Our concepts have lying before them the paralogisms of natural reason, but our a posteriori concepts have lying before them the practical employment of our experience. Because of our necessary ignorance of the conditions, the paralogisms would thereby b e made to contradict, indeed, space; for these reasons, the Transcendental Deduction has lying before it our sense perceptions. (Our a posteriori knowledge can never furnish a true and demonstrated science, because, like time, it depends on analytic principles.) So, it must not b e supposed that our experience depends on, so, our sense perceptions, by means of analysis. Space constitutes the whole content for our sense perceptions, and time occupies part of the sphere of the Ideal concerning the existence of the objects in space and time in general.\autocites{kant1999critique}

\printbibliography

\end{document}